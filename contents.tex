\section{Introduction}
\hspace{\parindent}Von Neumann and Morgenstern have developed a very fruitful theory of two-person zero-sum games in their book \emph{Theory of Games and Economic Behavior} \cite{1}. This books also contains a theory of $n$-person games of a type which we would call cooperative. This theory is based on an analysis of the interrelationships of the various coalitions which can be formed by the players of the game.

Our theory, in contradistinction, is based on the \emph{absence} of coalitions in that it is assumed that each participant acts independently, without collaboration or communication with any of the others.

The notion of an \emph{equilibrium} point is the basic ingredient in our theory. This notion yields a generalization of the concept of the solution of a two-person zero-sum game. It turns out that the set of equilibrium points of a two-person zero-sum game is simply the set of all pairs of opposing ``good strategies''.

In the immediately following sections we shall define equilibrium points and prove that a finite non-cooperative game always has at least one equilibrium point. We shall also introduce the notions of solvability and strong solvability of a non-cooperative game and prove a theorem on the geometrical structure of the set of equilibrium points of a solvable game.

As an example of the application of our theory we include a solution of a simplified three person poker game.

The motivation and interpretation of the mathematical concepts employed in the theory are reserved for discussion on a special section of this paper.

\newpage
\section{Formal Definitions and Terminology}
\hspace{\parindent}In this section we define the basic concepts of this paper and set up standard terminology and notation. Important definitions will be preceeded by a subtitle indicating the concept defined\footnote{We actually use \textbf{boldface} for definitions instead, but this note on the subtitles is left in to preserve the wording of the original.}. The non-cooperative idea will be implicit, rather than explicit, below.
\begin{definition}[Finite Game]
    For us an $\mathbf n$\textbf{-person game} will be a set of $n$ \textbf{players}, or \textbf{positions}, each with an associated finite of \textbf{pure strategies}; and corresponding to each player, $i$, a \textbf{pay-off function}, $P_i $, which maps the set of all $n$-tuples of pure strategies into the real numberes. When we use the word $\mathbf n$\textbf{-tuples} we shall always mean a set of $n$ items, with each item associated with a different player.
\end{definition}
\begin{definition}[Mixed Strategy, $S_i $]
   A \textbf{mixed strategy} of player $i$ will be a collection of non-negative numbers which have unit sum and are in one to one correspondence with his pure strategies.
\end{definition}
   We write $S_i=\sum_{\alpha }^{} c_{i\alpha }\pi_{i\alpha } $ with $\sum_{\alpha }^{} c_{i\alpha }=1 $ and $c_{i\alpha }\geq 0$ to represent such a mixed strategy, where the $\pi_{i\alpha }$'s are the pure strategies of player $i$. We regard the $S_i $'s as points in a simplex whose vertices are the $\pi_{i\alpha }$'s. This simplex may be regarded as a convex subset of a real vector space, giving us a natural process of linear combination for the mixed strategies.

   We shall use the suffixes 
